\documentclass{beamer}
\usepackage[utf8]{inputenc}
\usepackage[spanish]{babel}
\usetheme{metropolis}           % Use metropolis theme
\graphicspath{{images/}}

\usepackage{multicol}
\usepackage{listings}
\usepackage[default]{sourcesanspro}

\usepackage[scale=2]{ccicons}

\usepackage[
    type={CC},
    modifier={by-nc-sa},
    version={4.0},
]{doclicense}



\hypersetup{
    colorlinks=true,
    linkcolor=black,
    filecolor=magenta,
    urlcolor=cyan,
}

\title{Aprendizaje automático de un sistema interpretable de ayuda a la decisión para la estimación de la edad a partir de los huesos del pubis.}

\date{\today}
\author{\small Autor: Antonio David Villegas Yeguas \\ Directores: Óscar Cordón García y Sergio Damas Arroyo}
\institute[UGR]{Universidad de Granada\\
\medskip
\textit{advy99@correo.ugr.es}\\
\medskip
\url{https://github.com/advy99/TFG}
\doclicenseThis
}
\setbeamertemplate{caption}{\raggedright\insertcaption\par}

\begin{document}

 \maketitle

\begin{frame}{Índice}
\tableofcontents
\end{frame}




\section{Introducción y estado del arte del problema}
\begin{frame}{Estimación de la edad a partir de los huesos del pubis}

	\begin{columns}[T]
		\begin{column}{.4\textwidth}
			Problema complejo:
			\begin{itemize}
				\item No existe un método concreto.
				\item Trabajo subjetivo por parte de forenses.
				\item Falta de muestras para edades tempranas.
				\item Primera propuesta en 1920 por T. W. Todd \cite{todd}, clasificación en diez fases utilizando nueve características.
			\end{itemize}

		\end{column}

		\begin{column}{.7\textwidth}
			\begin{table}[H]
				\centering
				\resizebox{1\textwidth}{!}{%
					\begin{tabular}{|c|c|c|}
					\hline
					Crestas y surcos: Muy definidos & Superficie porosa irregular: Sí & Borde superior: Definido  \\ \hline
					\includegraphics[scale = 0.75]{huesos/crestas_surcos_muy_definidos.png}  &   \includegraphics[scale = 0.75]{huesos/superficie_porosa_si.png} &  \includegraphics[scale = 0.75]{huesos/borde_superior_definido.png}  \\ \hline
					Nódulo óseo: Presente & Borde inferior: No definido & Borde dorsal: Definido \\ \hline
					\includegraphics[scale = 0.75]{huesos/nodulo_oseo_presente.png} & \includegraphics[scale = 0.75]{huesos/borde_inferior_no_definido.png} &  \includegraphics[scale = 0.75]{huesos/borde_dorsal_definido.png} \\ \hline
					Plataforma dorsal: Presente & Bisel ventral: En proceso de formación & Borde ventral: Muchas excrecencias \\ \hline
					\includegraphics[scale = 0.75]{huesos/plataforma_dorsal_presente.png} & \includegraphics[scale = 0.75]{huesos/bisel_ventral_en_proceso_formacion.png} &   \includegraphics[scale = 0.75]{huesos/borde_ventral_muchas_excrecencias.png} \\ \hline
					\end{tabular}%
				}
				\caption{Algunos ejemplos de las características consideradas por Todd\cite{todd}.}\label{table:caracteristicas_todd}
			\end{table}
		\end{column}

	\end{columns}


\end{frame}

\begin{frame}[allowframebreaks]{Estado del arte}

\begin{itemize}
\item 1953 a 1973: Gilbert y McKern \cite{propuestaGilbert}. Cambio de enfoque a regresión, resultado expresado en un único valor y reducción de características.

\item 1990: J. M. Suchey y S. Brooks. Reducción del número de fases.

\item 2015: D. E. Slice y Algee-Hewitt. Escaneo de huesos utilizando Visión por Computador, clasificación con regresión lineal utilizando la propuesta de Suchey y Brooks. 41 muestras con una RECM de $17,15$ años.

\item 2015: D. Stoyanova, D. E. Slice y Algee-Hewitt. Modificación de las características utilizadas en su trabajo anterior. 57 muestras con una RECM de $19$ años.

\item 2018: A. Kotěrová, D. Navega, M. Štepanovský, Z. Buk, J. Brůžek y E. Cunha. Distintos modelos de aprendizaje, entre ellos regresión lineal, árboles de decisión y redes neuronales. 941 muestras, RECM de $12.1$ años y EAM de $9.7$ años.

\item 2021: J. C. Gámez-Granados, J. Irurita, R. Pérez, A. González, S. Damas, I. Alemán y O. Cordón (sometido a revista). Enfoque de Todd como un problema de clasificación ordinal, aprendizaje basado en reglas de forma iterativa. 892 muestras, RECM de $12,34$ años y 34 reglas fácilmente interpretables.

\end{itemize}

\end{frame}


\section{Objetivos}
\begin{frame}{Objetivos}
Nuestros objetivos serán:

\begin{enumerate}
	\item Obtener resultados sencillos, obtenidos a partir de modelos fácilmente interpretables.
	\item Tratar la alta dimensionalidad del problema.
	\item Estudiar el problema del número de muestras para resolver el problema.
\end{enumerate}

\end{frame}


\section{Inteligencia Artificial Explicable}
\begin{frame}{Inteligencia Artificial Explicable}

	Producir modelos fácilmente interpretables manteniendo buenos resultados y permitir que sean transparentes de cara a entender su funcionamiento.

	\begin{figure}[H]
		\centering
		\includegraphics[scale = 0.65]{esquema_xai.png}
		\caption{Diagrama de las ventajas de la Inteligencia Artificial Explicable desde varios puntos de vista. Imagen de \cite{XAI}}
		\label{fig:esquema_xai}
	\end{figure}

\end{frame}

\section{Propuesta e implementación}
\begin{frame}{Conjunto de datos}

	\begin{columns}[T]
		\begin{column}{.5\textwidth}
			\vspace*{1cm}
			\begin{enumerate}
				\item 892 muestras: 439 de la lateralidad izquierda y 453 de la lateralidad derecha.
				\item Clasificadas manualmente por el Laboratorio de Antropología Física de la UGR.
				\item Altamente desbalanceado.
			\end{enumerate}
		\end{column}

		\begin{column}{.5\textwidth}
			\begin{figure}[H]
				\centering
				\includegraphics[width=1.15\textwidth]{completo_regresion.csv.png}
				\caption{Densidad de cada valor el conjunto de datos completo.}
				\label{fig:conjunto_regresion}
			\end{figure}
		\end{column}

	\end{columns}

\end{frame}

\begin{frame}{Enfoque utilizado}

	\begin{itemize}
		\item Enfoque de Gilbert y McKern, pero sin suma final de valores y utilizando las características propuestas por Todd.
		\item Regresión simbólica.
		\item Programación Genética y GA-P para aprender expresiones.
		\item Selección de características por parte del modelo.
		\item Sobremuestreo de datos utilizando SMOGN.
	\end{itemize}

\end{frame}

\begin{frame}{Algoritmos: Sobremuestreo de datos}
	SMOGN:

	\begin{columns}[T]
		\begin{column}{.5\textwidth}
			\begin{itemize}
				\item Mejora de SMOTER.
				\item SMOTER: Adaptación de SMOTE para regresión.
				\item Generación datos sintéticos utilizando un k-nn.
				\item Introducción de ruido gaussiano.
				\item Submuestreo aleatorio.
				\item Tiene en cuenta si es seguro y fiable generar un nuevo dato sintético.
			\end{itemize}
		\end{column}

		\begin{column}{.5\textwidth}
			\begin{figure}[H]
			    \centering
				 \includegraphics[width=\textwidth]{SMOGN-5NN.png}
			    \caption{Ejemplo de un dato sintético generado por SMOGN utilizando los cinco vecinos más cercanos. Imagen obtenida de \cite{SMOGN}}
				 \label{fig:SMOGN-5NN}
			\end{figure}
		\end{column}

	\end{columns}

\end{frame}

\begin{frame}{Algoritmos: Programación Genética}

	\begin{columns}[T]

		\begin{column}{.4\textwidth}
			\begin{itemize}
				\item Propuesto por John Koza en 1990 \cite{kozaGP}.
				\item Algoritmo evolutivo.
				\item Uso de árboles en lugar de cromosomas.
				\item Distintos tipos de problemas.
				\item Resultado: Árbol que mejor se ajusta a los datos.
			\end{itemize}
		\end{column}


		\begin{column}{.6\textwidth}
			\begin{figure}[H]
			    \centering
				 \includegraphics[width=\textwidth]{generacional.png}
			    \caption{Esquema de un algoritmo evolutivo con un modelo generacional.}
				 \label{fig:modelo_generacioal}
			\end{figure}
		\end{column}

	\end{columns}

\end{frame}

\begin{frame}{Algoritmos: GA-P}
	\begin{itemize}
		\item Mejora de PG para regresión simbólica \cite{primerGAP}.
		\item Programación Genética para aprender expresiones y Algoritmo Genético para aprender constantes.
		\item Uso de nichos cuando las expresiones convergen.
	\end{itemize}

\end{frame}

\begin{frame}{Algoritmos: Validación de los resultados}
	Uso de 5x2-\textit{cross validation}:

	\begin{itemize}
		\item 5 particiones distintas de los datos al $50\%$.
		\item Utilizamos la primera mitad para entrenar y la segunda para validar. Realizamos una segunda iteración intercambiando las dos mitades.
		\item Se obtienen 10 valores, 2 por cada partición.
		\item Resultado final: Media de dichos diez valores.
	\end{itemize}

\end{frame}

\begin{frame}{Tecnologías utilizadas}


\end{frame}

\section{Análisis de resultados y comparación con el estado del arte}
\begin{frame}{Resultados}


\end{frame}

\begin{frame}{Importancia del sobremuestreo}


\end{frame}

\begin{frame}{Mejores expresiones obtenidas}


\end{frame}

\begin{frame}{Comparación con el estado del arte}


\end{frame}



\section{Conclusiones}
\begin{frame}{Conclusiones}


\end{frame}

\section{Bibliografía}
\begin{frame}[allowframebreaks]{Bibliografía}

\begin{thebibliography}{9}

	\bibitem{todd}

	\href{https://onlinelibrary.wiley.com/doi/abs/10.1002/ajpa.1330030301}{\scriptsize T. W. Todd, “Age changes in the pubic bone,” American Journal of Physical Anthropology, vol. 3, no. 3, pp. 285–328, 1920.}

	\bibitem{propuestaGilbert}

	\href{https://onlinelibrary.wiley.com/doi/abs/10.1002/ajpa.1330380109}{\scriptsize Gilbert, B. M., \& McKern, T. W. (1973). A method for aging the female os pubis. American Journal of Physical Anthropology, 38(1), 31-38.}

	\bibitem{XAI}

	\href{https://www.sciencedirect.com/science/article/pii/S1566253519308103}{\scriptsize Arrieta, A. B., Díaz-Rodríguez, N., Del Ser, J., Bennetot, A., Tabik, S., Barbado, A., ... \& Herrera, F. (2020). Explainable Artificial Intelligence (XAI): Concepts, taxonomies, opportunities and challenges toward responsible AI. Information Fusion, 58, 82-115.}

	\bibitem{SMOGN}

	\href{http://proceedings.mlr.press/v74/branco17a/branco17a.pdf}{\scriptsize Branco, P., Torgo, L., \& Ribeiro, R. P. (2017, October). SMOGN: a pre-processing approach for imbalanced regression. In First international workshop on learning with imbalanced domains: Theory and applications (pp. 36-50). PMLR.}

	\bibitem{kozaGP}

	\href{https://mitpress.mit.edu/books/genetic-programming}{\scriptsize Koza, J. R., \& Koza, J. R. (1992). Genetic programming: on the programming of computers by means of natural selection (Vol. 1). MIT press.}

	\bibitem{primerGAP}

	\href{https://ieeexplore.ieee.org/stamp/stamp.jsp?tp=&arnumber=393137}{\scriptsize Howard, L. M., \& D'Angelo, D. J. (1995). The GA-P: A genetic algorithm and genetic programming hybrid. IEEE expert, 10(3), 11-15.}

\end{thebibliography}


\end{frame}


\section{Preguntas}


\end{document}
