\section{Estado del arte}

El problema de la estimación de la edad a partir de restos óseos es un problema que ya se ha tratado en distintas ocasiones, desde modificaciones a la propuesta de Todd, propuestas de como automatizar la estimación utilizando las características que proponía Todd, hasta estudios que utilizaban técnicas de visión por computador para recoger y procesar las características de los restos.

Uno de los más relevantes es un estudio publicado en el año 2015 en \textit{Journal of forensic sciences} \cite{modelandoHuesos3D}, en el que se escaneaban los restos tomados como muestras para la variación de la sínfisis púbica y con esta variación realizar la estimación de la edad utilizando un modelo de regresión lineal. El conjunto de datos utilizado en este experimento se compone de 41 esqueletos de personas estadounidenses y logran obtener una raíz del error cuadrático medio de unos 17.15 años.

Ese mismo año, los mismos autores presentaron una mejora \cite{mejoraModelandoHuesos3D} en la que, en lugar de utilizar la variación total de la sínfisis púbica, utilizaban la flexión de un plano de forma que dicho plano coincida con la superficie del hueso. De esta forma, utilizando un conjunto de datos similar a su experimento anterior y la curvatura de la sínfisis púbica entrenaron un modelo de regresión lineal con el que obtuvieron una raíz del error cuadrático medio de unos 19 años.


Más adelante, en 2018, varios investigadores de la República Checa publicaron un trabajo \cite{estimacionHuesosCadera} en el que, con un conjunto de 941 restos óseos de personas de distinta raza entre 19 y 100 años, utilizando datos de los huesos de la cadera estudiaban las carácterísticas comunes y las que diferenciaban las distintas edades, y consideraban 9 modelos distintos para realizar una estimación, desde un sistema de puntiación tradicional, utilizando la variación entre los huesos, distintos de regresión lineal, árboles de decisiones o redes neuronales artificiales. De estos modelos se llega a la conclusión que el mejor es un modelo de regresión multilineal, con el que obtienen una raíz del error cuadrático medio de unos 12.1 años.
