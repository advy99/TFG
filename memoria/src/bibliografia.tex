\addcontentsline{toc}{section}{Referencias}

\begin{thebibliography}{9}


	\bibitem{todd}

	\href{https://onlinelibrary.wiley.com/doi/abs/10.1002/ajpa.1330030301}{T. W. Todd, “Age changes in the pubic bone,” American Journal of Physical Anthropology, vol. 3, no. 3, pp. 285–328, 1920.}



	\bibitem{XAI}

	\href{https://www.sciencedirect.com/science/article/pii/S1566253519308103?via%3Dihub}{Arrieta, A. B., Díaz-Rodríguez, N., Del Ser, J., Bennetot, A., Tabik, S., Barbado, A., ... \& Herrera, F. (2020). Explainable Artificial Intelligence (XAI): Concepts, taxonomies, opportunities and challenges toward responsible AI. Information Fusion, 58, 82-115.}


	\bibitem{laboratorioForenseUGR}

	\href{https://livemetrics.ugr.es/laboratorio-singular/antropologia-forense/}{Página web del Laboratorio de Antropología Física de la Universidad de Granada.}

	\bibitem{sucheyBrooks}

	\href{https://link.springer.com/article/10.1007/BF02437238}{Brooks, S., \& Suchey, J. M. (1990). Skeletal age determination based on the os pubis: a comparison of the Acsádi-Nemeskéri and Suchey-Brooks methods. Human evolution, 5(3), 227-238.}

	\bibitem{modelandoHuesos3D}

	\href{https://onlinelibrary.wiley.com/doi/full/10.1111/1556-4029.12778}{Slice, D. E., \& Algee‐Hewitt, B. F. (2015). Modeling bone surface morphology: a fully quantitative method for age‐at‐death estimation using the pubic symphysis. Journal of forensic sciences, 60(4), 835-843.}

	\bibitem{mejoraModelandoHuesos3D}

	\href{https://onlinelibrary.wiley.com/doi/full/10.1002/ajpa.22797}{Stoyanova, D., Algee‐Hewitt, B. F., \& Slice, D. E. (2015). An enhanced computational method for age‐at‐death estimation based on the pubic symphysis using 3 D laser scans and thin plate splines. American journal of physical anthropology, 158(3), 431-440.}


	\bibitem{componentBased}

	\href{https://www.sciencedirect.com/science/article/pii/S0379073815003254}{Dudzik, B., \& Langley, N. R. (2015). Estimating age from the pubic symphysis: A new component-based system. Forensic science international, 257, 98-105.}

	\bibitem{estimacionHuesosCadera}

	\href{https://www.sciencedirect.com/science/article/pii/S0379073818301440}{Kotěrová, A., Navega, D., Štepanovský, M., Buk, Z., Brůžek, J., \& Cunha, E. (2018). Age estimation of adult human remains from hip bones using advanced methods. Forensic science international, 287, 163-175.}


	\bibitem{fuzzyAgeEstimation}

	\href{https://ieeexplore.ieee.org/abstract/document/8015760?casa_token=1WPuf-NM3_wAAAAA:uZKFy6fFA-yPGnoET1qhbVjGvlpa0XfVr9_ONKGulSATUCC8NimIp-sYNIE6l7RNaKj7z0tDGQ}{Villar, P., Alemán, I., Castillo, L., Damas, S., \& Cordón, O. (2017, July). A first approach to a fuzzy classification system for age estimation based on the pubic bone. In 2017 IEEE International Conference on Fuzzy Systems (FUZZ-IEEE) (pp. 1-6). IEEE.}

	\bibitem{NSLVOrdAge}

	Gámez-Granados, J. C., Irurita, J., Pérez, R., González, A., Damas, S., Alemán, I. \& Cordón, O. Automating Todd’s Age Estimation Method from the Pubic Bone with Explainable Machine Learning

	\bibitem{NSLVOrd}

	\href{https://www.sciencedirect.com/science/article/pii/S0888613X16300706}{Gámez, J. C., García, D., González, A., \& Pérez, R. (2016). Ordinal classification based on the sequential covering strategy. International Journal of Approximate Reasoning, 76, 96-110.}

	\bibitem{kozaGP}

	\href{https://mitpress.mit.edu/books/genetic-programming}{Koza, J. R., \& Koza, J. R. (1992). Genetic programming: on the programming of computers by means of natural selection (Vol. 1). MIT press.}

	\bibitem{PGgramaticas}

	\href{https://www.researchgate.net/profile/Pa-Whigham/publication/2450222_Grammatically-based_Genetic_Programming/links/55c3c89908aebc967df1b765/Grammatically-based-Genetic-Programming.pdf}{Whigham, P. A. (1995, July). Grammatically-based genetic programming. In Proceedings of the workshop on genetic programming: from theory to real-world applications (Vol. 16, No. 3, pp. 33-41).}

	\bibitem{primerGAP}

	\href{https://ieeexplore.ieee.org/stamp/stamp.jsp?tp=&arnumber=393137}{Howard, L. M., \& D'Angelo, D. J. (1995). The GA-P: A genetic algorithm and genetic programming hybrid. IEEE expert, 10(3), 11-15.}

	\bibitem{GAPredElectrica}

	\href{https://link.springer.com/article/10.1023/A:1008384630089}{Cordón, O., Herrera, F., \& Sánchez, L. (1999). Solving electrical distribution problems using hybrid evolutionary data analysis techniques. Applied Intelligence, 10(1), 5-24.}


	\bibitem{PGregresionSimbolica}

	\href{https://ieeexplore.ieee.org/document/889734}{Augusto, D. A., \& Barbosa, H. J. (2000, November). Symbolic regression via genetic programming. In Proceedings. Vol. 1. Sixth Brazilian Symposium on Neural Networks (pp. 173-178). IEEE.}

	\bibitem{GAPFormulasBooleanas}

	\href{https://upcommons.upc.edu/handle/2099/3586}{Cordón García, O., Moya Anegón, F. D., \& Zarco Fernández, C. (2000). A GA-P algorithm to automatically formulate extended Boolean queries for a fuzzy information retrieval system. Mathware \& soft computing. 2000 Vol. 7 Núm. 2 [-3].}

\end{thebibliography}
