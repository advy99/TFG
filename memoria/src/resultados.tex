\section{Resultados}

\subsection{Número de ejecuciones y validación}

Para la ejecución de nuestros algoritmos separaremos los conjuntos de datos en dos partes, una de entrenamiento y otra de test, utilizando un $80\%$ del conjunto para el entrenamiento y el $20\%$ restante para el test.

De cara a validar los resultados, se ha utilizado la validación 5x2cv explicada con anterioridad sobre el conjunto de entrenamiento obtenido.

También tenemos que tener en cuenta que los algoritmos implementados son algoritmos estocásticos, es decir, la aleatoriedad juega una parte importante en las zonas del espacio de soluciones que se explorarán, por este motivo se han realizado cinco ejecuciones distintas, con cinco semillas aleatorias distintas.

\subsection{Parámetros utilizados}

Como vimos en el estado del arte, la Programación Genética es un algoritmo que necesita una población con un gran número de individuos, por este motivo se ha escogido una población de mil individuos. De cara a estudiar el sobreajuste de las expresiones y mantener el resultado simple se ha experimentado con distintas longitudes máximas de nodos, variando entre veinte, cuarenta y sesenta. Se ha tomado el valor inicial veinte ya que contamos con nueve nodos, y una expresión en la que intervengan todos los nodos con una única operación ya supondría una longitud de árbol de dicho tamaño.

Con respecto a los demás parámetros se han utilizado los más comunes en otras implementaciones de Programación Genética y que de forma empírica han funcionado mejor en nuestro conjunto de datos.

Se han utilizado los siguientes parámetros:

\begin{itemize}
	\item Tamaño de población: $1.000$.
	\item Probabilidad de que un nodo sea una variable: $30\%$.
	\item Tamaño máximo del árbol: $20$, $40$ y $60$.
	\item Número de evaluaciones: $1.000.000$.
	\item Probabilidad de cruce en PG: $75\%$.
	\item Probabilidad de cruce en GA (solo para GA-P): $75\%$.
	\item Probabilidad de mutación en PG: $5\%$.
	\item Probabilidad de mutación en GA (solo para GA-P): $5\%$.
	\item Probabilidad de cruce intra-nicho (solo para GA-P): $40\%$.
	\item Tamaño del torneo: $100$.
\end{itemize}

\newpage

\subsection{Resultados de Programación Genética en el conjunto de datos original}

Tras las distintas ejecuciones con las semillas aleatorias, en cada uno de los conjuntos de datos y con distintas profundidades máximas para las expresiones, obtenemos los siguientes resultados.


\subsubsection{Resultados en la lateralidad izquierda}

% Please add the following required packages to your document preamble:
% \usepackage{graphicx}
\begin{table}[H]
\centering
\resizebox{\textwidth}{!}{%
\begin{tabular}{|c|c|c|c|c|}
\hline
\multicolumn{5}{|c|}{\textbf{Produndidad máxima: 20}}                                                                        \\ \hline
\textbf{Semilla} & \textbf{ECM con 5x2-cv} & \textbf{RECM con 5x2cv} & \textbf{MAE con 5x2cv} & \textbf{Tiempo de ejecución (s)} \\ \hline
8324             & 79,9537                 & 8,9389                  & 7,46834                & 211,958                      \\ \hline
12345            & 80,3763                 & 8,9594                  & 7,5213                 & 211,641                      \\ \hline
34634            & 78,5994                 & 8,86191                 & 7,31334                & 212,611                      \\ \hline
34679            & 79,6023                 & 8,91243                 & 7,36373                & 210,634                      \\ \hline
92034            & 72,9244                 & 8,53155                 & 7,07619                & 211,236                      \\ \hline
\textbf{Media}   & \textbf{78,29122}       & \textbf{8,840838}       & \textbf{7,34858}       & \textbf{211,616}             \\ \hline
\end{tabular}%
}
\caption{Resultados de Programación Genética en la lateralidad izquierda con cinco semillas distintas y una profundidad máxima de 20 nodos.}\label{table:resultados_PG_l0_20}
\end{table}



% Please add the following required packages to your document preamble:
% \usepackage{graphicx}
\begin{table}[]
\centering
\resizebox{\textwidth}{!}{%
\begin{tabular}{|c|c|c|c|c|}
\hline
\multicolumn{5}{|c|}{\textbf{Produndidad máxima: 40}}                                                                        \\ \hline
\textbf{Semilla} & \textbf{ECM con 5x2-cv} & \textbf{RECM con 5x2cv} & \textbf{MAE con 5x2cv} & \textbf{Tiempo de ejecución (s)} \\ \hline
8324             & 77,4991                 & 8,79915                 & 7,35549                & 371,394                      \\ \hline
12345            & 113,413                 & 10,0316                 & 7,44315                & 376,57                       \\ \hline
34634            & 77,0115                 & 8,76665                 & 7,22587                & 367,263                      \\ \hline
34679            & 77,1488                 & 8,78025                 & 7,25083                & 376,047                      \\ \hline
92034            & 72,6009                 & 8,5159                  & 7,13994                & 377,999                      \\ \hline
\textbf{Media}   & \textbf{83,53466}       & \textbf{8,97871}        & \textbf{7,283056}      & \textbf{373,8546}            \\ \hline
\end{tabular}%
}
\caption{Resultados de Programación Genética en la lateralidad izquierda con cinco semillas distintas y una profundidad máxima de 40 nodos.}\label{table:resultados_PG_l0_40}

\end{table}

% Please add the following required packages to your document preamble:
% \usepackage{graphicx}
\begin{table}[H]
\centering
\resizebox{\textwidth}{!}{%
\begin{tabular}{|c|c|c|c|c|}
\hline
\multicolumn{5}{|c|}{\textbf{Produndidad máxima: 60}}                                                                        \\ \hline
\textbf{Semilla} & \textbf{ECM con 5x2-cv} & \textbf{RECM con 5x2cv} & \textbf{MAE con 5x2cv} & \textbf{Tiempo de ejecución (s)} \\ \hline
8324             & 76,673                  & 8,75413                 & 7,23217                & 522,304                      \\ \hline
12345            & 83,7292                 & 9,1046                  & 7,38679                & 516,579                      \\ \hline
34634            & 77,2803                 & 8,78271                 & 7,24127                & 519,752                      \\ \hline
34679            & 78,6668                 & 8,85517                 & 7,29779                & 523,821                      \\ \hline
92034            & 74,9574                 & 8,64884                 & 7,16493                & 517,639                      \\ \hline
\textbf{Media}   & \textbf{78,26134}       & \textbf{8,82909}        & \textbf{7,26459}       & \textbf{520,019}             \\ \hline
\end{tabular}%
}
\caption{Resultados de Programación Genética en la lateralidad izquierda con cinco semillas distintas y una profundidad máxima de 60 nodos.}\label{table:resultados_PG_l0_60}

\end{table}

\subsubsection{Resultados en la lateralidad derecha}


% Please add the following required packages to your document preamble:
% \usepackage{graphicx}
\begin{table}[H]
\centering
\resizebox{\textwidth}{!}{%
\begin{tabular}{|c|c|c|c|c|}
\hline
\multicolumn{5}{|c|}{\textbf{Produndidad máxima: 20}}                                                                            \\ \hline
\textbf{Semilla} & \textbf{ECM con 5x2-cv} & \textbf{RECM con 5x2cv} & \textbf{MAE con 5x2cv} & \textbf{Tiempo de ejecución (s)} \\ \hline
8324             & 72,3476                 & 8,50209                 & 6,99355                & 215,559                          \\ \hline
12345            & 70,884                  & 8,41339                 & 6,86307                & 213,978                          \\ \hline
34634            & 74,9666                 & 8,64992                 & 7,10458                & 211,862                          \\ \hline
34679            & 74,7413                 & 8,63821                 & 7,118                  & 210,305                          \\ \hline
92034            & 67,5138                 & 8,21472                 & 6,73837                & 213,598                          \\ \hline
\textbf{Media}   & \textbf{72,09066}       & \textbf{8,483666}       & \textbf{6,963514}      & \textbf{213,0604}                \\ \hline
\end{tabular}%
}
\caption{Resultados de Programación Genética en la lateralidad derecha con cinco semillas distintas y una profundidad máxima de 20 nodos.}\label{table:resultados_PG_l1_20}
\end{table}


% Please add the following required packages to your document preamble:
% \usepackage{graphicx}
\begin{table}[H]
\centering
\resizebox{\textwidth}{!}{%
\begin{tabular}{|c|c|c|c|c|}
\hline
\multicolumn{5}{|c|}{\textbf{Produndidad máxima: 40}}                                                                            \\ \hline
\textbf{Semilla} & \textbf{ECM con 5x2-cv} & \textbf{RECM con 5x2cv} & \textbf{MAE con 5x2cv} & \textbf{Tiempo de ejecución (s)} \\ \hline
8324             & 72,0551                 & 8,48533                 & 6,95435                & 388,36                           \\ \hline
12345            & 70,7415                 & 8,40321                 & 6,87752                & 391,053                          \\ \hline
34634            & 74,7811                 & 8,64441                 & 7,08979                & 390,666                          \\ \hline
34679            & 72,4849                 & 8,50504                 & 6,95179                & 389,966                          \\ \hline
92034            & 68,3543                 & 8,26171                 & 6,7361                 & 396,093                          \\ \hline
\textbf{Media}   & \textbf{71,68338}       & \textbf{8,45994}        & \textbf{6,92191}       & \textbf{391,2276}                \\ \hline
\end{tabular}%
}
\caption{Resultados de Programación Genética en la lateralidad derecha con cinco semillas distintas y una profundidad máxima de 40 nodos.}\label{table:resultados_PG_l1_40}

\end{table}


\begin{table}[H]
\centering
\resizebox{\textwidth}{!}{%
\begin{tabular}{|c|c|c|c|c|}
\hline
\multicolumn{5}{|c|}{\textbf{Produndidad máxima: 60}}                                                                            \\ \hline
\textbf{Semilla} & \textbf{ECM con 5x2-cv} & \textbf{RECM con 5x2cv} & \textbf{MAE con 5x2cv} & \textbf{Tiempo de ejecución (s)} \\ \hline
8324             & 77,6228                 & 8,79922                 & 7,10807                & 534,707                          \\ \hline
12345            & 68,9063                 & 8,29117                 & 6,73511                & 545,115                          \\ \hline
34634            & 74,2657                 & 8,61588                 & 7,11258                & 507,235                          \\ \hline
34679            & 87,2423                 & 9,10656                 & 7,10872                & 537,105                          \\ \hline
92034            & 71,1911                 & 8,42445                 & 6,81426                & 541,437                          \\ \hline
\textbf{Media}   & \textbf{75,84564}       & \textbf{8,647456}       & \textbf{6,975748}      & \textbf{533,1198}                \\ \hline
\end{tabular}%
}
\caption{Resultados de Programación Genética en la lateralidad derecha con cinco semillas distintas y una profundidad máxima de 60 nodos.}\label{table:resultados_PG_l1_60}

\end{table}

\subsubsection{Resultados en el conjunto completo}

% Please add the following required packages to your document preamble:
% \usepackage{graphicx}
\begin{table}[H]
\centering
\resizebox{\textwidth}{!}{%
\begin{tabular}{|c|c|c|c|c|}
\hline
\multicolumn{5}{|c|}{\textbf{Produndidad máxima: 20}}                                                                            \\ \hline
\textbf{Semilla} & \textbf{ECM con 5x2-cv} & \textbf{RECM con 5x2cv} & \textbf{MAE con 5x2cv} & \textbf{Tiempo de ejecución (s)} \\ \hline
8324             & 75,1256                 & 8,66525                 & 7,12232                & 344,445                          \\ \hline
12345            & 73,594                  & 8,57715                 & 7,08578                & 342,743                          \\ \hline
34634            & 73,0808                 & 8,54466                 & 7,01207                & 359,184                          \\ \hline
34679            & 75,1162                 & 8,66306                 & 7,18915                & 353,337                          \\ \hline
92034            & 70,5851                 & 8,39821                 & 6,92052                & 347,82                           \\ \hline
\textbf{Media}   & \textbf{73,50034}       & \textbf{8,569666}       & \textbf{7,065968}      & \textbf{349,5058}                \\ \hline
\end{tabular}%
}
\caption{Resultados de Programación Genética en el conjunto de datos completo con cinco semillas distintas y una profundidad máxima de 20 nodos.}\label{table:resultados_PG_c_20}
\end{table}



% Please add the following required packages to your document preamble:
% \usepackage{graphicx}
\begin{table}[H]
\centering
\resizebox{\textwidth}{!}{%
\begin{tabular}{|c|c|c|c|c|}
\hline
\multicolumn{5}{|c|}{\textbf{Produndidad máxima: 40}}                                                                            \\ \hline
\textbf{Semilla} & \textbf{ECM con 5x2-cv} & \textbf{RECM con 5x2cv} & \textbf{MAE con 5x2cv} & \textbf{Tiempo de ejecución (s)} \\ \hline
8324             & 75,368                  & 8,68014                 & 7,17403                & 665,789                          \\ \hline
12345            & 71,6442                 & 8,46147                 & 6,97584                & 652,943                          \\ \hline
34634            & 70,6075                 & 8,3956                  & 6,93068                & 660,407                          \\ \hline
34679            & 72,1347                 & 8,49127                 & 7,02663                & 683,528                          \\ \hline
92034            & 73,1519                 & 8,54904                 & 6,99538                & 651,94                           \\ \hline
\textbf{Media}   & \textbf{72,58126}       & \textbf{8,515504}       & \textbf{7,020512}      & \textbf{662,9214}                \\ \hline
\end{tabular}%
}
\caption{Resultados de Programación Genética en el conjunto de datos completo con cinco semillas distintas y una profundidad máxima de 40 nodos.}\label{table:resultados_PG_c_40}

\end{table}




% Please add the following required packages to your document preamble:
% \usepackage{graphicx}
\begin{table}[H]
\centering
\resizebox{\textwidth}{!}{%
\begin{tabular}{|c|c|c|c|c|}
\hline
\multicolumn{5}{|c|}{\textbf{Produndidad máxima: 20}}                                                                            \\ \hline
\textbf{Semilla} & \textbf{ECM con 5x2-cv} & \textbf{RECM con 5x2cv} & \textbf{MAE con 5x2cv} & \textbf{Tiempo de ejecución (s)} \\ \hline
8324             & 73,6171                 & 8,57348                 & 7,06422                & 949,263                          \\ \hline
12345            & 71,3675                 & 8,44683                 & 6,98279                & 951,941                          \\ \hline
34634            & 70,8811                 & 8,41334                 & 6,91272                & 940,346                          \\ \hline
34679            & 71,2928                 & 8,44038                 & 6,93763                & 921,818                          \\ \hline
92034            & 70,2425                 & 8,3779                  & 6,81619                & 937,515                          \\ \hline
\textbf{Media}   & \textbf{71,4802}        & \textbf{8,450386}       & \textbf{6,94271}       & \textbf{940,1766}                \\ \hline
\end{tabular}%
}
\caption{Resultados de Programación Genética en el conjunto de datos completo con cinco semillas distintas y una profundidad máxima de 60 nodos.}\label{table:resultados_PG_c_60}

\end{table}


\subsection{Resultados de GA-P en el conjunto de datos original}



\subsubsection{Resultados en la lateralidad izquierda}



% Please add the following required packages to your document preamble:
% \usepackage{graphicx}
\begin{table}[H]
\centering
\resizebox{\textwidth}{!}{%
\begin{tabular}{|c|c|c|c|c|}
\hline
\multicolumn{5}{|c|}{\textbf{Produndidad máxima: 20}}                                                                            \\ \hline
\textbf{Semilla} & \textbf{ECM con 5x2-cv} & \textbf{RECM con 5x2cv} & \textbf{MAE con 5x2cv} & \textbf{Tiempo de ejecución (s)} \\ \hline
8324             & 80,6048                 & 8,97422                 & 7,53073                & 282,953                          \\ \hline
12345            & 77,9568                 & 8,82054                 & 7,29926                & 285,879                          \\ \hline
34634            & 76,5017                 & 8,74331                 & 7,2109                 & 281,017                          \\ \hline
34679            & 79,9467                 & 8,93532                 & 7,40111                & 286,302                          \\ \hline
92034            & 73,4627                 & 8,5659                  & 7,11149                & 281,452                          \\ \hline
\textbf{Media}   & \textbf{77,69454}       & \textbf{8,807858}       & \textbf{7,310698}      & \textbf{283,5206}                \\ \hline
\end{tabular}%
}
\caption{Resultados de GA-P en la lateralidad izquierda con cinco semillas distintas y una profundidad máxima de 20 nodos.}\label{table:resultados_GAP_l0_20}
\end{table}




% Please add the following required packages to your document preamble:
% \usepackage{graphicx}
\begin{table}[H]
\centering
\resizebox{\textwidth}{!}{%
\begin{tabular}{|c|c|c|c|c|}
\hline
\multicolumn{5}{|c|}{\textbf{Produndidad máxima: 40}}                                                                            \\ \hline
\textbf{Semilla} & \textbf{ECM con 5x2-cv} & \textbf{RECM con 5x2cv} & \textbf{MAE con 5x2cv} & \textbf{Tiempo de ejecución (s)} \\ \hline
8324             & 77,6193                 & 8,80512                 & 7,33588                & 490,818                          \\ \hline
12345            & 75,9628                 & 8,71163                 & 7,2485                 & 501,287                          \\ \hline
34634            & 75,4656                 & 8,6828                  & 7,18711                & 500,661                          \\ \hline
34679            & 76,5455                 & 8,73873                 & 7,22826                & 502,369                          \\ \hline
92034            & 73,7506                 & 8,58473                 & 7,13092                & 500,513                          \\ \hline
\textbf{Media}   & \textbf{75,86876}       & \textbf{8,704602}       & \textbf{7,226134}      & \textbf{499,1296}                \\ \hline
\end{tabular}%
}
\caption{Resultados de GA-P en la lateralidad izquierda con cinco semillas distintas y una profundidad máxima de 40 nodos.}\label{table:resultados_GAP_l0_40}
\end{table}




\begin{table}[H]
\centering
\resizebox{\textwidth}{!}{%
\begin{tabular}{|c|c|c|c|c|}
\hline
\multicolumn{5}{|c|}{\textbf{Produndidad máxima: 60}}                                                                            \\ \hline
\textbf{Semilla} & \textbf{ECM con 5x2-cv} & \textbf{RECM con 5x2cv} & \textbf{MAE con 5x2cv} & \textbf{Tiempo de ejecución (s)} \\ \hline
8324             & 78,8322                 & 8,87626                 & 7,42958                & 694,155                          \\ \hline
12345            & 78,0311                 & 8,81658                 & 7,30498                & 674,094                          \\ \hline
34634            & 82,1144                 & 9,02879                 & 7,31266                & 693,09                           \\ \hline
34679            & 75,9406                 & 8,70831                 & 7,16713                & 690,594                          \\ \hline
92034            & 75,8019                 & 8,70209                 & 7,19296                & 689,885                          \\ \hline
\textbf{Media}   & \textbf{78,14404}       & \textbf{8,826406}       & \textbf{7,281462}      & \textbf{688,3636}                \\ \hline
\end{tabular}%
}
\caption{Resultados de GA-P en la lateralidad izquierda con cinco semillas distintas y una profundidad máxima de 60 nodos.}\label{table:resultados_GAP_l0_60}

\end{table}


\subsubsection{Resultados en la lateralidad derecha}



% Please add the following required packages to your document preamble:
% \usepackage{graphicx}
\begin{table}[H]
\centering
\resizebox{\textwidth}{!}{%
\begin{tabular}{|c|c|c|c|c|}
\hline
\multicolumn{5}{|c|}{\textbf{Produndidad máxima: 20}}                                                                            \\ \hline
\textbf{Semilla} & \textbf{ECM con 5x2-cv} & \textbf{RECM con 5x2cv} & \textbf{MAE con 5x2cv} & \textbf{Tiempo de ejecución (s)} \\ \hline
8324             & 71,9909                 & 8,47982                 & 7,03758                & 283,665                          \\ \hline
12345            & 71,0153                 & 8,42255                 & 6,85738                & 286,86                           \\ \hline
34634            & 73,6482                 & 8,57357                 & 7,06444                & 283,782                          \\ \hline
34679            & 71,439                  & 8,44593                 & 6,98865                & 282,896                          \\ \hline
92034            & 67,4192                 & 8,20595                 & 6,7038                 & 281,093                          \\ \hline
\textbf{Media}   & \textbf{71,10252}       & \textbf{8,425564}       & \textbf{6,93037}       & \textbf{283,6592}                \\ \hline
\end{tabular}%
}
\caption{Resultados de GA-P en la lateralidad derecha con cinco semillas distintas y una profundidad máxima de 20 nodos.}\label{table:resultados_GAP_l1_20}

\end{table}


% Please add the following required packages to your document preamble:
% \usepackage{graphicx}
\begin{table}[H]
\centering
\resizebox{\textwidth}{!}{%
\begin{tabular}{|c|c|c|c|c|}
\hline
\multicolumn{5}{|c|}{\textbf{Produndidad máxima: 40}}                                                                            \\ \hline
\textbf{Semilla} & \textbf{ECM con 5x2-cv} & \textbf{RECM con 5x2cv} & \textbf{MAE con 5x2cv} & \textbf{Tiempo de ejecución (s)} \\ \hline
8324             & 72,508                  & 8,5066                  & 6,94798                & 517,619                          \\ \hline
12345            & 71,2989                 & 8,43973                 & 6,81754                & 521,668                          \\ \hline
34634            & 75,3864                 & 8,67228                 & 7,18904                & 515,871                          \\ \hline
34679            & 71,2897                 & 8,4386                  & 6,94084                & 523,88                           \\ \hline
92034            & 67,7882                 & 8,22882                 & 6,70779                & 510,017                          \\ \hline
\textbf{Media}   & \textbf{71,65424}       & \textbf{8,457206}       & \textbf{6,920638}      & \textbf{517,811}                 \\ \hline
\end{tabular}%
}
\caption{Resultados de GA-P en la lateralidad derecha con cinco semillas distintas y una profundidad máxima de 40 nodos.}\label{table:resultados_GAP_l1_40}

\end{table}

% Please add the following required packages to your document preamble:
% \usepackage{graphicx}
\begin{table}[H]
\centering
\resizebox{\textwidth}{!}{%
\begin{tabular}{|c|c|c|c|c|}
\hline
\multicolumn{5}{|c|}{\textbf{Produndidad máxima: 60}}                                                                            \\ \hline
\textbf{Semilla} & \textbf{ECM con 5x2-cv} & \textbf{RECM con 5x2cv} & \textbf{MAE con 5x2cv} & \textbf{Tiempo de ejecución (s)} \\ \hline
8324             & 71,8235                 & 8,474                   & 6,95127                & 719,636                          \\ \hline
12345            & 71,9868                 & 8,47804                 & 6,918                  & 691,101                          \\ \hline
34634            & 74,9277                 & 8,65065                 & 7,11936                & 700,404                          \\ \hline
34679            & 71,3174                 & 8,43537                 & 6,90571                & 709,719                          \\ \hline
92034            & 70,6022                 & 8,39437                 & 6,82546                & 709,367                          \\ \hline
\textbf{Media}   & \textbf{72,13152}       & \textbf{8,486486}       & \textbf{6,94396}       & \textbf{706,0454}                \\ \hline
\end{tabular}%
}
\caption{Resultados de GA-P en la lateralidad derecha con cinco semillas distintas y una profundidad máxima de 60 nodos.}\label{table:resultados_GAP_l1_60}

\end{table}


\subsubsection{Resultados en el conjunto completo}



% Please add the following required packages to your document preamble:
% \usepackage{graphicx}
\begin{table}[H]
\centering
\resizebox{\textwidth}{!}{%
\begin{tabular}{|c|c|c|c|c|}
\hline
\multicolumn{5}{|c|}{\textbf{Produndidad máxima: 20}}                                                                            \\ \hline
\textbf{Semilla} & \textbf{ECM con 5x2-cv} & \textbf{RECM con 5x2cv} & \textbf{MAE con 5x2cv} & \textbf{Tiempo de ejecución (s)} \\ \hline
8324             & 78,2469                 & 8,83818                 & 7,34737                & 466,672                          \\ \hline
12345            & 72,5627                 & 8,51482                 & 7,05337                & 468,713                          \\ \hline
34634            & 72,6187                 & 8,5183                  & 7,07125                & 443,885                          \\ \hline
34679            & 74,6101                 & 8,63406                 & 7,17107                & 456,268                          \\ \hline
92034            & 72,2163                 & 8,4945                  & 6,96785                & 465,966                          \\ \hline
\textbf{Media}   & \textbf{74,05094}       & \textbf{8,599972}       & \textbf{7,122182}      & \textbf{460,3008}                \\ \hline
\end{tabular}%
}
\caption{Resultados de GA-P en el conjunto de datos con cinco semillas distintas y una profundidad máxima de 20 nodos.}\label{table:resultados_GAP_c_20}
\end{table}



% Please add the following required packages to your document preamble:
% \usepackage{graphicx}
\begin{table}[H]
\centering
\resizebox{\textwidth}{!}{%
\begin{tabular}{|c|c|c|c|c|}
\hline
\multicolumn{5}{|c|}{\textbf{Produndidad máxima: 40}}                                                                            \\ \hline
\textbf{Semilla} & \textbf{ECM con 5x2-cv} & \textbf{RECM con 5x2cv} & \textbf{MAE con 5x2cv} & \textbf{Tiempo de ejecución (s)} \\ \hline
8324             & 76,0559                 & 8,71923                 & 7,21849                & 881,488                          \\ \hline
12345            & 72,2559                 & 8,49909                 & 7,05527                & 878,581                          \\ \hline
34634            & 70,0585                 & 8,3691                  & 6,90598                & 879,425                          \\ \hline
34679            & 73,1284                 & 8,5493                  & 7,05026                & 839,744                          \\ \hline
92034            & 70,8889                 & 8,41686                 & 6,88382                & 899,355                          \\ \hline
\textbf{Media}   & \textbf{72,47752}       & \textbf{8,510716}       & \textbf{7,022764}      & \textbf{875,7186}                \\ \hline
\end{tabular}%
}
\caption{Resultados de GA-P en el conjunto de datos con cinco semillas distintas y una profundidad máxima de 40 nodos.}\label{table:resultados_GAP_c_40}
\end{table}

% Please add the following required packages to your document preamble:
% \usepackage{graphicx}
\begin{table}[H]
\centering
\resizebox{\textwidth}{!}{%
\begin{tabular}{|c|c|c|c|c|}
\hline
\multicolumn{5}{|c|}{\textbf{Produndidad máxima: 20}}                                                                            \\ \hline
\textbf{Semilla} & \textbf{ECM con 5x2-cv} & \textbf{RECM con 5x2cv} & \textbf{MAE con 5x2cv} & \textbf{Tiempo de ejecución (s)} \\ \hline
8324             & 75,5747                 & 8,68849                 & 7,1571                 & 1208,32                          \\ \hline
12345            & 73,3194                 & 8,55511                 & 7,02977                & 1216,75                          \\ \hline
34634            & 69,9813                 & 8,3638                  & 6,88685                & 1235,84                          \\ \hline
34679            & 72,1627                 & 8,49306                 & 7,03382                & 1238,75                          \\ \hline
92034            & 70,1123                 & 8,3699                  & 6,84281                & 1214,96                          \\ \hline
\textbf{Media}   & \textbf{72,23008}       & \textbf{8,494072}       & \textbf{6,99007}       & \textbf{1222,924}                \\ \hline
\end{tabular}%
}
\caption{Resultados de GA-P en el conjunto de datos con cinco semillas distintas y una profundidad máxima de 60 nodos.}\label{table:resultados_GAP_c_60}

\end{table}



\subsection{Resultados de Programación Genética con datos sobremuestreados}

\subsubsection{Resultados en la lateralidad izquierda}



\subsubsection{Resultados en la lateralidad derecha}


\subsubsection{Resultados en el conjunto completo}


\subsection{Resultados de GA-P con datos sobremuestreados}

\subsubsection{Resultados en la lateralidad izquierda}


\subsubsection{Resultados en la lateralidad derecha}


\subsubsection{Resultados en el conjunto completo}


\newpage
