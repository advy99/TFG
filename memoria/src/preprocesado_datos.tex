\section{Preprocesado de los datos}

Nuestro conjunto de datos se compone de tres ficheros:

\begin{enumerate}
	\item \texttt{lateralidad0.arff} : 439 muestras de la lateralidad izquierda del pubis.
	\item \texttt{lateralidad1.arff} : 453 muestras de la lateralidad derecha del pubis.
	\item \texttt{completo.arff} : 892 muestas de ambas lateralidades, se compone de los dos ficheros antes en conjunto.
\end{enumerate}

\begin{figure}[H]
	\begin{lstlisting}[language={}]
	NoGrooves,Absence,Defined,Absent,Defined,Present,Absent,Absent,FormedWithoutRarefactions,Ph07-35-39
	\end{lstlisting}
	\caption{Ejemplo de dato perteneciente a la fase 7 (entre 35 y 39 años) propuesta por Todd, del conjunto de datos \texttt{completo.arff}.}
	\label{fig:ejemplo_dato}
\end{figure}

Como vemos en la figura \ref{fig:ejemplo_dato} los datos tienen asignados valores categóricos para cada característica, y finalmente la fase a la que pertenecen.

\subsection{Transformación de las características}

De cara a poder trabajar con estos datos para aplicar el sobremuestreo a los datos y después regresión simbólica tenemos que transformar estos valores categóricos a valores numéricos.

Para esto simplemente leeremos los valores y a cada característica le asignaremos un valor entre uno y el máximo número de posibles valores para dicha característica, de forma que una característica con dos posibles valores será $1$ o $2$, dependiendo de cual de los dos valores aparezca.


\begin{table}[H]
\resizebox{\textwidth}{!}{%
\begin{tabular}{|c|c|c|}
\hline
                                                      & \textbf{Valor categórico}       & \textbf{Valor numérico asignado} \\ \hline
\multirow{6}{*}{\textbf{Crestas y surcos}}            & No hay surcos                   & 1                                \\ \cline{2-3}
                                                      & Restos de surcos                & 2                                \\ \cline{2-3}
                                                      & Surcos poco profundas           & 3                                \\ \cline{2-3}
                                                      & Crestas en formación            & 4                                \\ \cline{2-3}
                                                      & Crestas poco profundas          & 5                                \\ \cline{2-3}
                                                      & Crestas con porosidad regular   & 6                                \\ \hline
\multirow{3}{*}{\textbf{Superficie porosa irregular}} & No                              & 1                                \\ \cline{2-3}
                                                      & Medianamente                    & 2                                \\ \cline{2-3}
                                                      & Si                              & 3                                \\ \hline
\multirow{2}{*}{\textbf{Borde superior}}              & No definido                     & 1                                \\ \cline{2-3}
                                                      & Definido                        & 2                                \\ \hline
\multirow{2}{*}{\textbf{Nódulo óseo}}                 & Ausente                         & 1                                \\ \cline{2-3}
                                                      & Presente                        & 2                                \\ \hline
\multirow{2}{*}{\textbf{Borde inferior}}              & No definido                     & 1                                \\ \cline{2-3}
                                                      & Definido                        & 2                                \\ \hline
\multirow{2}{*}{\textbf{Borde dorsal}}                & No definido                     & 1                                \\ \cline{2-3}
                                                      & Definido                        & 2                                \\ \hline
\multirow{2}{*}{\textbf{Plataforma dorsal}}           & Ausente                         & 1                                \\ \cline{2-3}
                                                      & Presente                        & 2                                \\ \hline
\multirow{3}{*}{\textbf{Bisel ventral}}               & Ausente                         & 1                                \\ \cline{2-3}
                                                      & En proceso de formación         & 2                                \\ \cline{2-3}
                                                      & Formado                         & 3                                \\ \hline
\multirow{5}{*}{\textbf{Borde ventral}}               & Ausente                         & 1                                \\ \cline{2-3}
                                                      & Parcialmente formado            & 2                                \\ \cline{2-3}
                                                      & Formado sin excrecencias        & 3                                \\ \cline{2-3}
                                                      & Formado con pocas excrecencias  & 4                                \\ \cline{2-3}
                                                      & Formado con muchas excrecencias & 5                                \\ \hline
\end{tabular}%
}
\caption{Transformaciones aplicadas a las características antes de realizar el sobremuestreo.}\label{table:transformaciones_numericas}

\end{table}

Para transformar las etiquetas simplemente se ha hecho la media del rango de edades de cada fase, quedando de la siguiente forma:


\begin{table}[H]
\centering
\resizebox{\textwidth}{!}{%
\begin{tabular}{|c|c|c|c|c|c|c|c|c|c|c|}
\hline
\textbf{Fase}                  & Fase 1 & Fase 2 & Fase 3 & Fase 3 & Fase 3 & Fase 6 & Fase 6 & Fase 8 & Fase 9 & Fase 10 \\ \hline
\textbf{Valor correspondiente} & 19     & 20.5   & 23     & 25.5   & 25.5   & 32.5   & 37     & 42     & 47     & 55      \\ \hline
\end{tabular}%
}
\end{table}

Esta transformación es necesaria ya que los algoritmos a utilizar darán como resultado un valor numérico, que será la edad estimada. Como vemos en principio no hay problema, ya que las fases con mayor rango de edades contienen dos años de variación con respecto al valor asignado, a excepción de la última fase ya que esta es todos los mayores de 50. Para solucionar el problema de la última fase se ha asignado un valor que continua con la tendencia creciente del valor, aumentándolo en siete con respecto a la fase anterior, un valor mayor que la diferencia entre las dos fases anteriores (cuya diferencia es de cinco).

\subsection{Algoritmo Borderline-SMOTE}

El algoritmo Borderline-SMOTE \cite{BL-SMOTE} se basa en una variante de SMOTE \cite{SMOTE}.

SMOTE se trata de un método para conseguir nuevos datos de las clases minoritarias de forma sintética.

Las clases minoritarias se sobremuestrean tomando cada muestra de dicha clase minoritaria, e introduciendo valores sintéticos a los largo del segmento que une a todos o a cualquiera de los $k$ vecinos más cercanos de la clase. Dependiendo de la cantidad de datos sintéticos a generar se escogerá el valor de $k$ para seleccionar más o menos vecinos.


\begin{figure}[H]
	\centering
	\includegraphics[scale = 1]{ej_smote.png}
	\caption{Comparación de SMOTE con otros métodos de sobremuestreo en el conjunto de datos Phonome. En el eje X los falsos positivos y en el eje Y los verdaderos positivos. Imagen obtenida de \cite{SMOTE}.}
	\label{fig:comparacionSMOTE}
\end{figure}

Como vemos en la figura \ref{fig:comparacionSMOTE}, este método se comporta bastante bien en comparación con otros métodos y es capaz de obtener buenos resultados, sin embargo, tiene algunos problemas.

Uno de estos problemas es que cuando existe ruido entre las clases, o las clases se encuentran dispersas por el espacio de valores, gran parte de los valores sintéticos se encontrarán en una zona del espacio que no corresponde a su clase.

Este problema se hace mucho más presente en nuestro caso, al contar con tantas clases y con tantas características, y para resolverlo utilizaremos la variante Borderline-SMOTE.

Esta variante, basada en SMOTE, propone buscar los límites en el espacio de valores de la clase a obtener nuevos datos sintéticos, de forma que cuando se generen dichos datos nuevos, estén dentro de dichos límites. De esta forma evitamos problemas con conjuntos de datos con mucho ruido y múltiples clases.

En su artículo proponen dos variantes, una donde las nuevas muestras sintéticas son creadas a partir de todo el conjunto de datos, y otra donde las nuevas muestras solo se generan a partir de los datos considerados en el límite.


\begin{figure}[H]
    \centering
    \begin{subfigure}[b]{0.33\textwidth}
		  \includegraphics[width=\textwidth]{bl-smote-original.png}
        \caption{}
        \label{fig:blSMOTE-orig}
    \end{subfigure}
    \begin{subfigure}[b]{0.33\textwidth}
        \includegraphics[width=\textwidth]{bl-smote-datos-borderline.png}
        \caption{}
        \label{fig:blSMOTE-border}
    \end{subfigure}
    \begin{subfigure}[b]{0.33\textwidth}
        \includegraphics[width=\textwidth]{bl-smote-datos-sinteticos.png}
        \caption{}
        \label{fig:blSMOTE-sintetico}
    \end{subfigure}

    \caption{\ref{fig:blSMOTE-orig} Conjunto de datos original. \ref{fig:blSMOTE-border} datos que conforman el límite de la clase minoritaria (cuadros azules rellenos). \ref{fig:blSMOTE-sintetico} Nuevos datos generados de forma sintética dentro de los límites de la clase (cuadros azules no rellenos). Imagen obtenida de \cite{BL-SMOTE}.}
	 \label{fig:ejemploBL-SMOTE}

\end{figure}


\begin{figure}[H]
    \centering
	 \begin{subfigure}[b]{\textwidth}
		 \centering
		 \includegraphics[width=0.8\textwidth]{resampling_smote.png}
		 \caption{A la izquierda bordes de las clases, y a la derecha sobremuestreo con SMOTE}
		 \label{fig:SMOTE-cmp}
	 \end{subfigure}

    \begin{subfigure}[b]{\textwidth}
		 \centering
		  \includegraphics[width=0.8\textwidth]{resampling_blsmote.png}
        \caption{A la izquierda bordes de las clases, y a la derecha sobremuestreo con BorderlineSMOTE}
        \label{fig:BLSMOTE-cmp}
    \end{subfigure}

    \caption{Comparación de SMOTE y BorderlineSMOTE para un conjunto de datos generado aleatoriamente.}\label{fig:BLSMOTE-SMOTE}

\end{figure}

Esta claro que, como vemos en la figura \ref{fig:ejemploBL-SMOTE} y la figura \ref{fig:BLSMOTE-SMOTE}, esta técnica es mucho más versátil y conveniente que SMOTE, motivo por el que la utilizaremos.


De cara a su implementación se utilizará el módulo de Python \texttt{imbalanced-learn} que implementa el algoritmo de BorderlineSMOTE \cite{imblearnBLSMOTE}.


\subsection{Reducción de la dimensionalidad}

Tras aplicar la transformación numérica y el sobremuestreo simplemente sumamos los valores numéricos de las característica para quedarnos con un fichero de datos con dos columnas, el valor final de la suma, y la correspondiente etiqueta.

\newpage

\subsection{Resultado tras el preprocesado}

Tras aplicar todas las fases del preprocesado obtenemos los conjuntos de datos finales que utilizaremos.

\begin{figure}[H]
    \centering
	  \includegraphics[width=0.6\textwidth]{conjunto_datos/num_elementos_fase_l0.png}
     \label{fig:l0-orig}
    \caption{Conteo de elementos por fase en el conjunto de datos original de la lateralidad izquierda.}

\end{figure}

\begin{figure}[H]
    \centering
     \includegraphics[width=0.6\textwidth]{conjunto_datos/num_elementos_fase_l0_BL-SMOTE.png}
     \label{fig:l0-over}
    \caption{Conteo de elementos por fase en el conjunto de datos de la lateralidad izquierda tras aplicar BorderlineSMOTE.}
\end{figure}

\begin{figure}[H]
    \centering
	  \includegraphics[width=0.6\textwidth]{conjunto_datos/num_elementos_fase_l1.png}
     \label{fig:l1-orig}
    \caption{Conteo de elementos por fase en el conjunto de datos original de la lateralidad derecha.}

\end{figure}

\begin{figure}[H]
    \centering
     \includegraphics[width=0.6\textwidth]{conjunto_datos/num_elementos_fase_l1_BL-SMOTE.png}
     \label{fig:l1-over}
    \caption{Conteo de elementos por fase en el conjunto de datos de la lateralidad derecha tras aplicar BorderlineSMOTE.}
\end{figure}



\begin{figure}[H]
    \centering
	  \includegraphics[width=0.6\textwidth]{conjunto_datos/num_elementos_fase_completo.png}
     \label{fig:l1-orig}
    \caption{Conteo de elementos por fase en el conjunto de datos completo original.}

\end{figure}

\begin{figure}[H]
    \centering
     \includegraphics[width=0.6\textwidth]{conjunto_datos/num_elementos_fase_completo_BL-SMOTE.png}
     \label{fig:l1-over}
    \caption{Conteo de elementos por fase en el conjunto de datos completo tras aplicar BorderlineSMOTE.}
\end{figure}
