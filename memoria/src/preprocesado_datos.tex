\section{Preprocesado de los datos}

Nuestro conjunto de datos se compone de tres ficheros:

\begin{enumerate}
	\item \texttt{lateralidad0.arff} : 439 muestras de la lateralidad izquierda del pubis.
	\item \texttt{lateralidad1.arff} : 453 muestras de la lateralidad derecha del pubis.
	\item \texttt{completo.arff} : 892 muestas de ambas lateralidades, se compone de los dos ficheros antes en conjunto.
\end{enumerate}

\begin{figure}[H]
	\begin{lstlisting}[language={}]
	NoGrooves,Absence,Defined,Absent,Defined,Present,Absent,Absent,FormedWithoutRarefactions,36
	\end{lstlisting}
	\caption{Ejemplo de un dato cuya edad de muerte es 36 años, del conjunto de datos \texttt{completo.arff}.}
	\label{fig:ejemplo_dato}
\end{figure}

Como vemos en la figura \ref{fig:ejemplo_dato} los datos tienen asignados valores categóricos para cada característica, y finalmente la edad a la que murió la persona con las características asociadas.

\subsection{Transformación de las características}

De cara a poder trabajar con estos datos para aplicar el sobremuestreo a los datos y después regresión simbólica tenemos que transformar estos valores categóricos a valores numéricos.

Para esto simplemente leeremos los valores y a cada característica le asignaremos un valor entre uno y el máximo número de posibles valores para dicha característica, de forma que una característica con dos posibles valores será $1$ o $2$, dependiendo de cual de los dos valores aparezca.


\begin{table}[H]
\centering
\resizebox{\textwidth}{!}{%
\begin{tabular}{|c|c|c|c|}
\hline
                                                      & \textbf{Variable asignada} & \textbf{Valor categórico}       & \textbf{Valor numérico asignado} \\ \hline
\multirow{6}{*}{\textbf{Crestas y surcos}}            & \multirow{6}{*}{$x_0$}     & No hay surcos                   & 1                                \\ \cline{3-4}
                                                      &                            & Restos de surcos                & 2                                \\ \cline{3-4}
                                                      &                            & Surcos poco profundas           & 3                                \\ \cline{3-4}
                                                      &                            & Crestas en formación            & 4                                \\ \cline{3-4}
                                                      &                            & Crestas poco profundas          & 5                                \\ \cline{3-4}
                                                      &                            & Crestas con porosidad regular   & 6                                \\ \hline
\multirow{3}{*}{\textbf{Superficie porosa irregular}} & \multirow{3}{*}{$x_1$}     & No                              & 1                                \\ \cline{3-4}
                                                      &                            & Medianamente                    & 2                                \\ \cline{3-4}
                                                      &                            & Si                              & 3                                \\ \hline
\multirow{2}{*}{\textbf{Borde superior}}              & \multirow{2}{*}{$x_2$}     & No definido                     & 1                                \\ \cline{3-4}
                                                      &                            & Definido                        & 2                                \\ \hline
\multirow{2}{*}{\textbf{Nódulo óseo}}                 & \multirow{2}{*}{$x_3$}     & Ausente                         & 1                                \\ \cline{3-4}
                                                      &                            & Presente                        & 2                                \\ \hline
\multirow{2}{*}{\textbf{Borde inferior}}              & \multirow{2}{*}{$x_4$}     & No definido                     & 1                                \\ \cline{3-4}
                                                      &                            & Definido                        & 2                                \\ \hline
\multirow{2}{*}{\textbf{Borde dorsal}}                & \multirow{2}{*}{$x_5$}     & No definido                     & 1                                \\ \cline{3-4}
                                                      &                            & Definido                        & 2                                \\ \hline
\multirow{2}{*}{\textbf{Plataforma dorsal}}           & \multirow{2}{*}{$x_6$}     & Ausente                         & 1                                \\ \cline{3-4}
                                                      &                            & Presente                        & 2                                \\ \hline
\multirow{3}{*}{\textbf{Bisel ventral}}               & \multirow{3}{*}{$x_7$}     & Ausente                         & 1                                \\ \cline{3-4}
                                                      &                            & En proceso de formación         & 2                                \\ \cline{3-4}
                                                      &                            & Formado                         & 3                                \\ \hline
\multirow{5}{*}{\textbf{Borde ventral}}               & \multirow{5}{*}{$x_8$}     & Ausente                         & 1                                \\ \cline{3-4}
                                                      &                            & Parcialmente formado            & 2                                \\ \cline{3-4}
                                                      &                            & Formado sin excrecencias        & 3                                \\ \cline{3-4}
                                                      &                            & Formado con pocas excrecencias  & 4                                \\ \cline{3-4}
                                                      &                            & Formado con muchas excrecencias & 5                                \\ \hline
\end{tabular}%
}
\caption{Transformaciones aplicadas a las características antes de realizar el sobremuestreo.}\label{table:transformaciones_numericas}

\end{table}

Con respecto a las etiquetas, aunque en un principio este problema esté pensado para clasificación y se divida en diez fases, contamos con el conjunto de datos preparado para regresión, es decir, con un valor numérico asignado.

% TODO : A ver que hago con esta parte de Borderline-SMOTE, en principio la voy a dejar por si me sirve para la segunda parte de clasificación

\subsection{Sobremuestreo}

% TODO: cambiar cita
El algoritmo que utilizaremos, SMOGN \cite{SMOGN}, se basa en distintas variantes de SMOTE \cite{SMOTE}, donde se mejora el algoritmo original y se adapta a regresión.

\subsubsection{SMOTE}

SMOTE se trata de un método para conseguir balancear el número de datos para un problema de clasificación creando nuevos datos de las clases minoritarias de forma sintética.

Las clases minoritarias se sobremuestrean tomando cada muestra de dicha clase minoritaria, e introduciendo valores sintéticos a lo largo del segmento que une a todos o a cualquiera de los $k$ vecinos más cercanos de la clase. Dependiendo de la cantidad de datos sintéticos a generar se escogerá el valor de $k$ para seleccionar más o menos vecinos.


\begin{figure}[H]
	\centering
	\includegraphics[scale = 1]{ej_smote.png}
	\caption{Comparación de SMOTE con otros métodos de sobremuestreo en el conjunto de datos Phonome. En el eje X los falsos positivos y en el eje Y los verdaderos positivos. Imagen obtenida de \cite{SMOTE}.}
	\label{fig:comparacionSMOTE}
\end{figure}

Como vemos en la figura \ref{fig:comparacionSMOTE}, este método se comporta bastante bien en comparación con otros métodos y es capaz de obtener buenos resultados, sin embargo, tiene algunos problemas.

Uno de estos problemas es que cuando existe ruido entre las clases, o las clases se encuentran dispersas por el espacio de valores, gran parte de los valores sintéticos se encontrarán en una zona del espacio que no corresponde a su clase.

Este problema se hace mucho más presente en nuestro caso, al contar con tantas clases y con tantas características, y para resolverlo se propuso la variante Borderline-SMOTE.

\subsubsection{Borderline SMOTE}


Borderline SMOTE \cite{BL-SMOTE} se trata de una variante de SMOTE que propone buscar los límites en el espacio de valores de la clase a obtener nuevos datos sintéticos, de forma que cuando se generen dichos datos nuevos, estén dentro de dichos límites. De esta forma evitamos problemas con conjuntos de datos con mucho ruido y múltiples clases.

En su artículo proponen dos variantes, una donde las nuevas muestras sintéticas son creadas a partir de todo el conjunto de datos, y otra donde las nuevas muestras solo se generan a partir de los datos considerados en el límite.


\begin{figure}[H]
    \centering
    \begin{subfigure}[b]{0.33\textwidth}
		  \includegraphics[width=\textwidth]{bl-smote-original.png}
        \caption{}
        \label{fig:blSMOTE-orig}
    \end{subfigure}
    \begin{subfigure}[b]{0.33\textwidth}
        \includegraphics[width=\textwidth]{bl-smote-datos-borderline.png}
        \caption{}
        \label{fig:blSMOTE-border}
    \end{subfigure}
    \begin{subfigure}[b]{0.33\textwidth}
        \includegraphics[width=\textwidth]{bl-smote-datos-sinteticos.png}
        \caption{}
        \label{fig:blSMOTE-sintetico}
    \end{subfigure}

    \caption{\ref{fig:blSMOTE-orig} Conjunto de datos original. \ref{fig:blSMOTE-border} datos que conforman el límite de la clase minoritaria (cuadros azules rellenos). \ref{fig:blSMOTE-sintetico} Nuevos datos generados de forma sintética dentro de los límites de la clase (cuadros azules no rellenos). Imagen obtenida de \cite{BL-SMOTE}.}
	 \label{fig:ejemploBL-SMOTE}

\end{figure}


\begin{figure}[H]
    \centering
	 \begin{subfigure}[b]{\textwidth}
		 \centering
		 \includegraphics[width=0.8\textwidth]{resampling_smote.png}
		 \caption{A la izquierda bordes de las clases, y a la derecha sobremuestreo con SMOTE}
		 \label{fig:SMOTE-cmp}
	 \end{subfigure}

    \begin{subfigure}[b]{\textwidth}
		 \centering
		  \includegraphics[width=0.8\textwidth]{resampling_blsmote.png}
        \caption{A la izquierda bordes de las clases, y a la derecha sobremuestreo con BorderlineSMOTE}
        \label{fig:BLSMOTE-cmp}
    \end{subfigure}

    \caption{Comparación de SMOTE y BorderlineSMOTE para un conjunto de datos generado aleatoriamente.}\label{fig:BLSMOTE-SMOTE}

\end{figure}

Esta claro que, como vemos en la figura \ref{fig:ejemploBL-SMOTE} y la figura \ref{fig:BLSMOTE-SMOTE}, esta técnica es mucho más versátil y conveniente que SMOTE.


\subsubsection{SMOTER: SMOTE para Regresión}

Uno de los principales problemas que tiene SMOTE es que solo es posible aplicarlo para problemas de clasificación. Para solucionar este problema en 2013 investigadores de la Universidad de Oporto y de Waikato proponen una adaptación del algoritmo SMOTE (o sus variantes) para regresión \cite{SMOTER}.

Este trabajo se centra en resolver tres problemas para poder adaptar SMOTE a un problema de regresión:

\begin{enumerate}
	\item Como definir los datos relevantes y los datos que se consideran raros.
	\item Como crear los nuevos datos sintéticos.
	\item Como asignar un valor numérico a esos datos sintéticos para usarlos en un modelo de regresión.
\end{enumerate}

Con respecto al primer problema, proponen una función de relevancia, en el que el usuario introduzca información sobre que etiquetas hay que obtener nuevos datos, ya que al ser un problema de regresión el número de posibles valores a sobremuestrear es infinito de cara al algoritmo, necesita algo de información sobre que etiquetas es necesario obtener los nuevos datos.

Para resolver el segundo problema proponen generar los datos sintéticos de forma similar al algoritmo SMOTE, adaptando algunos detalles para que sea capaz de trabajar con regresión.

Por último, para tratar el tercer problema, se propone utilizar una media ponderada utilizando el valor de rareza de las etiquetas de los datos con los que se han generado los datos sintéticos.

\begin{figure}[H]
	\centering
	\includegraphics[scale = 1.4]{smoter_resultados.png}
	\caption{Comparación entre un modelo entrenado con los datos originales o los datos sobremuestreados con SMOTER en diecisiete conjuntos de datos. Imagen obtenida de \cite{SMOTER}.}
	\label{fig:smoter_resultados}
\end{figure}

Como vemos, en la mayoría de conjuntos de datos es capaz de mejorar notablemente los resultados, y en ninguna de las pruebas se obtienen peores resultados que en el conjunto de datos original, por lo que vemos que este algoritmo es una buena opción para obtener nuevos datos sintéticos en regresión.

\subsubsection{Algoritmo a utilizar: SMOGN}




\newpage

\subsection{Resultado tras el preprocesado}

Tras aplicar todas las fases del preprocesado obtenemos los conjuntos de datos finales que utilizaremos.

\begin{figure}[H]
    \centering
	  \includegraphics[width=0.6\textwidth]{conjunto_datos/num_elementos_fase_l0.png}
    \caption{Conteo de elementos por fase en el conjunto de datos original de la lateralidad izquierda.}
	 \label{fig:l0-orig}
\end{figure}

\begin{figure}[H]
    \centering
     \includegraphics[width=0.6\textwidth]{conjunto_datos/num_elementos_fase_l0_BL-SMOTE.png}
    \caption{Conteo de elementos por fase en el conjunto de datos de la lateralidad izquierda tras aplicar BorderlineSMOTE.}
	 \label{fig:l0-over}
\end{figure}

\begin{figure}[H]
    \centering
	  \includegraphics[width=0.6\textwidth]{conjunto_datos/num_elementos_fase_l1.png}
    \caption{Conteo de elementos por fase en el conjunto de datos original de la lateralidad derecha.}
	 \label{fig:l1-orig}
\end{figure}

\begin{figure}[H]
    \centering
     \includegraphics[width=0.6\textwidth]{conjunto_datos/num_elementos_fase_l1_BL-SMOTE.png}
    \caption{Conteo de elementos por fase en el conjunto de datos de la lateralidad derecha tras aplicar BorderlineSMOTE.}
	 \label{fig:l1-over}
\end{figure}



\begin{figure}[H]
    \centering
	  \includegraphics[width=0.6\textwidth]{conjunto_datos/num_elementos_fase_completo.png}
     \label{fig:completo-orig}
    \caption{Conteo de elementos por fase en el conjunto de datos completo original.}

\end{figure}

\begin{figure}[H]
    \centering
     \includegraphics[width=0.6\textwidth]{conjunto_datos/num_elementos_fase_completo_BL-SMOTE.png}
    \caption{Conteo de elementos por fase en el conjunto de datos completo tras aplicar BorderlineSMOTE.}
	 \label{fig:completo-over}
\end{figure}


Como vemos en las figuras \ref{fig:l0-over}, \ref{fig:l1-over} y \ref{fig:completo-over} ya tenemos conjuntos de datos balanceado y, aplicando la propuesta de Gilbert, con una única variable que serán los conjuntos de datos que utilizaremos.
