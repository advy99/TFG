\section{GA-P}

GA-P se trata de una mejora de Programación Genética. Esta propuesta fue publicada en 1995 por investigadores de la Universidad de Georgia \cite{primerGAP}.

Esta modificación está pensada especialmente para problemas donde Programación Genética se utiliza para regresión simbólica.

\subsection{Como funciona GA-P}

Uno de los principales problemas de Programación Genética es el aprendizaje de las constantes numéricas. GA-P soluciona este problema siendo un híbrido entre Programación Genética y Algoritmo Genético.

GA-P, además de utilizar la representación en árbol de Programación Genética, añadirá un cromosoma de forma que las constantes numéricas en el árbol se referirán a posiciones en dicho cromosoma. De esta forma GA-P, además de ajustar el árbol, también ajustará el cromosoma como si fuera un problema de Programación Genética para así buscar unos mejores valores de las constantes numéricas sin tener que realizar operaciones con estas para obtener cierto valor.

\begin{figure}[H]
    \centering
	  \includegraphics[width=0.6\textwidth]{expresion_gap.png}
    \caption{Ejemplo de una expresión de GA-P con un cromosoma asociado.}
	 \label{fig:expresion_gap}
\end{figure}

De esta forma, como vemos en la figura \ref{fig:expresion_gap}, los nodos de las expresiones tendrán asociados variables del problema (como $x$ en el ejemplo), o constantes, las cuales tendrá un valor asociado en el cromosoma de la expresión. Por ejemplo, con los valores actuales de la imagen la expresión sería la siguiente:

$$ -6 \cdot ( (-3.9 / x) + 12 ) $$

Esta representación nos permitirá entrenar la parte de Programación Genética del modelo con los operadores vistos en Programación Genética, y por otra parte ajustar las constantes numéricas utilizando el cromosoma, la parte de Algoritmo Genético.

\subsection{Operadores de GA-P}

Además de los operadores de la parte de Programación Genética, que se mantendrán, GA-P utilizará operadores para la parte de Algoritmo Genético. En este caso se podrían aplicar todo tipo de operadores de codificación real ya que existe una gran variedad, pero se comentarán los operadores más comunes y que utilizaremos en este caso.

\subsubsection{Operador de cruce}

Como operador de cruce utilizaremos el cruce BLX-$\alpha$, propuesto en 1993 por Larry J. Eshelman y J. David Schaffer \cite{cruceBLXalfa}.

Este operador de cruce se trata de una modificación del operador de cruce para cromosomas de codificación real de Radcliffe \cite{cruceRadcliffe}, en el que para cada punto de ambos cromosomas se escogía un valor aleatorio escogido de forma uniforme entre ambos valores reales.

BLX-$\alpha$ además de tomar en cuenta el intervalo entre ambos números de ambos cromosomas $I$, amplía dicho intervalo en un $\alpha \cdot I$, es decir, si al cruzar dos cromosomas, si el valor del padre es $x$ y el de la madre es $y$, $I = y - x$, utilizando el operador de cruce de Radcliffe el nuevo valor estaría en el intervalo $[x, y]$, mientras que utilizando el cruce BLX-$\alpha$ utilizará como intervalo $[x - I \cdot \alpha, y + I \cdot \alpha]$. Como vemos, BLX-$0.0$ sería equivalente al operador de cruce de Radcliffe.

Con esta modificación conseguimos que el operador de cruce sea capaz de tener una mayor exploración del espacio de búsqueda, además de evitar que los cromosomas converjan demasiado rápido.

\begin{figure}[H]
	 \centering
	 \includegraphics[width=0.6\textwidth]{blxalfa.png}
	 \caption{Ejemplo del intervalo al que aplicar BLX-$\alpha$.}
	\label{fig:cruceBLXa}
\end{figure}


\subsubsection{Operador de mutación}

Como operador de mutación utilizaremos el operador de mutación no uniforme de Michalewicz \cite{mutacionMichalewicz}.

Esta mutación se realiza de la siguiente forma:

\[ x^{t + 1} =
	\begin{cases}
		x_k^t + \Delta(t, k, x_k)\\
		x_k^t - \Delta(t, k, x_k)
	\end{cases}
\]

Donde $x_k^t$ es el valor del posición $k$ del cromosoma en la generación $t$, y la función $\Delta$ es la siguiente:

$$ \Delta(t, y) = y \cdot r \cdot (1 - \frac{t}{T})^b $$

Donde $r$ es un número aleatorio en el intervalo $[0, 1]$, $T$ es el número total de generaciones, y $b$ es un parámetro interno con el que ajustar la no linealidad.

Con esta mutación conseguimos un operador de mutación que sea brusco en la etapa inicial del algoritmo, añadiendo una mayor diversidad en el cromosoma, mientras que en las últimas generaciones no será tan brusco, comportamiento que buscamos ya que al final del algoritmo queremos que se realice una mayor explotación de la mejor solución encontrada.

\subsection{Problemas de GA-P}

Uno de los problemas que podemos encontrar en GA-P es que a priori no podemos saber si una expresión es una buena solución sin ajustar las constantes numéricas que intervienen en dicha expresión.

Esto, sumado a la gran diversidad que añade el operador de cruce de la parte de Programación Genética, puede hacer que pasemos por alto una zona del espacio de búsqueda con buenas soluciones, pero que al no tener suficientemente desarrollados los cromosomas parezcan malas soluciones.


\newpage

\subsection{Modificaciones de GA-P: GA-P con nichos}

De cara a resolver este problema, en el año 2000 investigadores de la Universidad de Granada y la Universitat Politècnica de Catalunya propusieron un nuevo esquema de GA-P donde utilizar nichos \cite{GAPnichos}. Esta propuesta se basa en utilizar nichos, de forma que con cierta probabilidad se decidirá si hacer un cruce intra-nicho o inter-nicho. Diremos que dos expresiones están en el mismo nicho cuando la forma de su expresión sea la misma.

De esta forma, los cruces funcionan de la siguiente forma:

\begin{enumerate}
	\item Cruce intra-nicho: Si se decide realizar un cruce intra-nicho, este cruce escogerá como padres dos elementos en el mismo nicho y, como ambos tienen la misma expresión, solo se cruzará el cromosoma, la parte de Algoritmo Genético, de forma que se intente mejorar las constantes numéricas de las expresiones.
	\item Cruce inter-nicho: Se escogerán padres de dos nichos distintos, y el cruce será el cruce clásico de GA-P, donde se intenta mejorar tanto la expresión como el cromosoma.
\end{enumerate}


Con esta modificación conseguiremos un mejor ajuste de la parte de Algoritmo Genético en las expresiones.


\newpage
