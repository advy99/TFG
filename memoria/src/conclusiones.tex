\section{Conclusiones}

En este trabajo hemos introducido un problema real y de gran interés que se lleva investigando durante un largo periodo de tiempo. Hemos revisado su estado del arte para entender el punto en el que se encuentra el problema, así como los distintos métodos y enfoques que se han propuesto para resolverlo. Tras esto, se ha propuesto utilizar un tipo de algoritmo evolutivo que nos permite obtener resultados fácilmente interpretables además de una introducción general a los distintos tipos de algoritmos evolutivos, explicado en profundidad la variante concreta que hemos utilizado así como distintas propuestas y mejoras sobre esta. Hemos explicado el funcionamiento de regresión simbólica aplicada al problema y realizado un estudio y aplicación de técnicas de sobremuestreo de datos para poder mejorar el conjunto de datos utilizado debido al gran desafío de obtener un conjunto de datos completo para este problema.

Más en detalle, se ha realizado un estudio del problema de la estimación de la edad a partir de los huesos del pubis, revisando la propuesta original de Thomas Wingate Todd, así como distintos enfoques que surgieron a partir de esta propuesta, como los de Gilbert y McKern, llegando a la conclusión de que se utilizará la propuesta de Gilbert y McKern de cara a enfocar el problema como un problema de regresión pero utilizando las características propuestas por Todd de cara a no eliminar grados de libertad en las características. Tras discutir la importancia de resolver este problema en el ámbito forense, se ha realizado una introducción a la inteligencia artificial explicable (XAI) y porqué en este problema en concreto el encontrar una solución interpretable es de gran interés, permitiendo a los forenses encontrar nuevas relaciones entre las características del pubis para poder mejorar su trabajo.

Tras esto, se ha discutido sobre la dificultad de encontrar un buen conjunto de datos que represente a todas las posibles edades de cara a poder obtener un modelo que sea capaz de generalizar los resultados sin importar la entrada y la necesidad de aplicarlo sobre nuestro conjunto de datos, llegando a analizar técnicas de sobremuestreo, centrándonos en SMOTE y sus variantes debido a sus buenos resultados en el estado del arte, así como sus adaptaciones a regresión debido al enfoque que se le ha dado al problema en este trabajo.

Por último se ha presentado una propuesta de algoritmo evolutivo cuyos resultados sean fácilmente interpretables y se pueda aplicar regresión simbólica con el fin de no solo aprender las constantes para ajustar la expresión, sino para aprender también la forma de la expresión, Programación Genética. Se ha realizado una introducción a algoritmos evolutivos, así como a las distintas variantes existentes, explicado en profundidad el funcionamiento de Programación Genética, la utilizada en este trabajo, sus ventajas y desventajas, además de una variante para mejorar su funcionamiento en regresión simbólica, GA-P, haciendo un híbrido entre Programación Genética y un Algoritmo Genético.

Tras realizar los distintos experimentos sobre estos dos algoritmos, tanto con el conjunto de datos original como el conjunto con datos generados de forma sintética, hemos concluido que, aunque ciertas características propuestas por Todd no influyen en la estimación de la edad y corroborando este hecho con el estado del arte, el enfoque tomado ha sido un acierto, llegando a igualar los resultados del estado del arte con el conjunto de datos original y mejorando dichos resultados utilizando el conjunto con datos sintéticos obteniendo fórmulas simples y fácilmente interpretables.
