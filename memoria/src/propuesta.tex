\section{Propuesta}

\subsection{Conjunto de datos a utilizar}

Para realizar este trabajo utilizaremos un conjunto de datos con $892$ muestras de la sínfisis púbica clasificados manualmente por investigadores del Laboratorio de Antropología Física de la Universidad de Granada utilizando las diez fases propuestas por Todd.

Este conjunto de datos se divide en dos partes, las muestras tomadas de la lateralidad izquierda ($439$ muestras) y la lateralidad derecha ($453$ muestras) del pubis.

\subsubsection{Problema del balanceo de clases}

Como vimos en la figura \ref{fig:conteo_original} con el conteo de cada una de las fases, los datos de este problema se encuentran altamente desbalanceados, hay fases en las que apenas contamos con datos, y la diferencia entre número de datos en una fase baja y una fase alta es muy grande. De cara a resolver este problema proponemos utilizar un algoritmo de sobremuestreo de datos, al igual que  \cite{NSLVOrdAge}.

En este artículo se proponen distintas técnicas de sobremuestreo, principalmente utilizando un sobremuestreo de forma aleatoria, utilizando el algoritmo SMOTE \cite{revisionSMOTE} (y varias variaciones de este algoritmo), y utilizando el algoritmo ADASYN \cite{propuestaADASYN}.

En nuestro caso, al utilizar el mismo conjunto de datos, podemos ver en su artículo que la técnica que mejor ha funcionado para este conjunto de datos es Borderline-SMOTE. Por este motivo, en este trabajo también utilizaremos este algoritmo.

\subsubsection{Problema de la alta dimensionalidad}

Para resolver este problema queremos obtener soluciones simples y fácilmente interpretables. Un problema a la hora de conseguir este objetivo es que el conjunto de datos se compone de un alto número de características, con múltiples valores para dichas características en algunos casos.

De cara a solucionar esto proponemos utilizar la modificación propuesta en 1973 por Gilbert y McKern \cite{propuestaGilbert} al sistema de clasificación de Todd. En esta publicación, para realizar la estimación de la edad, a cada característica se le asocia un valor numérico. De esta forma realizar la estimación de la edad se basa en, para cada caso, sumar la puntuación asociada a sus característica y dependiendo del rango del valor final se asignará la fase.

\subsection{Algoritmos a utilizar}

\subsection{Planificación temporal}
