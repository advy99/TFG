\clearpage
\begin{center}
	{\large\textbf{\thetitle}}


	\theauthor
\end{center}

\textbf{Palabras clave:} Antropología Forense, Estimación de la Edad a partir de Restos Óseos, Inteligencia Artificial Explicable,
Aprendizaje Automático, Programación Genética, Metaheurísticas.

\textbf{Resumen:}

La estimación de la edad, tanto de personas fallecidas como vivas, es una de las tareas más importantes en la antropología forense. Actualmente este proceso de estimación de la edad se basa en analizar ciertas características como la apariencia, morfología o patrones de osificación en los restos óseos de los individuos. Entre los distintos métodos utilizados destaca la sínfisis púbica debido a su alta fiabilidad. Este proceso se realiza de forma manual, aunque en los últimos años han aparecido diversos trabajos que intentan automatizar la estimación de la edad a partir de restos óseos. Los resultados de estos métodos son demasiado complejos para servir de apoyo a los forenses de cara a mejorar y perfeccionar la técnica, a pesar de conseguir unas estimaciones aceptables.

Por este motivo, uno de los últimos enfoques aplicados sugiere buscar métodos de inteligencia artificial explicable, permitiendo obtener modelos sencillos y fácilmente entendibles. Esto puede conllevar una peor estimación por parte del modelo, pero el objetivo final no es automatizar por completo la tarea, si no servir de apoyo al forense y de esta forma facilitar su trabajo a la vez que se puede llegar a encontrar detalles que antes no se han tenido en cuenta en el proceso de estimación de la edad.

En este trabajo se partirá de la propuesta de Thomas Wingate Todd en el año 1920, en el que proponía una forma de automatizar la clasificación de restos en diez fases de edad, de cara a trabajar con el conjunto de datos de características de sínfisis púbica del laboratorio de Antropología Física de la Universidad de Granada.

Este problema se trata de un problema complejo de aprendizaje automático debido a que el tamaño del conjunto de datos es muy pequeño, tiene una gran cantidad de características y está altamente desbalanceado. Sumado a esto, tenemos como objetivo que el modelo final sea simple de cara a dar con un buen resultado que los forenses sean capaces de aplicar.

Para cumplir estos objetivos, en este trabajo se tratarán diversas técnicas para balancear los datos, reducir su dimensionalidad y se tratarán diversos enfoques de algoritmos evolutivos de cara a obtener un modelo simple.

Nuestro trabajo se centrará en utilizar algoritmos basados en expresiones matemáticas de cara a, con una sola fórmula, estimar la edad de cada individuo, primero explorando y entrenando el enfoque clásico de Programación Genética, mejorarlo añadiendo un algoritmo genético a la Programación Genética (GA-P) con el objetivo de suplir algunos de sus problemas, así como realizar un estudio de algoritmos de sobremuestreo para mejorar el conjunto de datos a utilizar.

\newpage


\begin{center}
	{\large\textbf{\thetitleEN}}


	\theauthor
\end{center}

\textbf{Keywords:} Forensic anthropology, Age estimation from bone remains, Explainable Artificial Intelligence,
Machine Learning, Genetic Programming, Metaheuristics.

\textbf{Abstract:}

Age estimation, both for deceased and living people, is one of the most important tasks for forensic anthropology. Currently this process of age estimation is based on analyzing certain characteristics such as appearence, morphology or ossification patterns in the skeletal remains of individuals. Among the different methods used, the pubic symphysis stands out due to its high reliability. This process is performed manually, although in recent years there have been several studies that attempt to automate age estimation from bone remains. The results of these methods are too complex to support forensic scientists at improving and refining the technique, despite achieving acceptable estimates.

For this reason, one of the latest applied approaches suggests searching for explainable artificial intelligence methods, allowing to obtain simple and easily understandable models. This may lead to a worse estimation by the model, but the ultimate goal is not to fully automate the task, but to support the forensic scientists and thus facilitate his work while finding details that have not been previously taken into account in the age estimation process.

In this work we will start from Thomas Wingate Todd's proposal in 1920, in which he proposed a way to automate the classification of remains in ten age stages, in order to work with the dataset of pubic symphysis characteristics of the Physical Anthropology Laboratory of the University of Granada.

This problem is a complex machine learning problem beacuse the dataset size is very small, has a large amount of features and is highly unbalanced. In addition to this, we aim to keep the final model simple in order to give a good result that forensic scientists will be able to apply.


To meet these objectives, this work will discuss different techniques to balance the data, reduce its dimensionality and discuss various evolutionary algorithm approaches in order to obtain a simple model.


To begin with, we will use mathematical expresssion-based algorithms to estimate the age of each individual with a single formula, first exploring and training the classical Genetic Programming approach, improving it by adding a genetic algorithm to Genetic Programming (GA-P) in order to overcome some of its problems, and finally using grammar-based systems to obtain easily interpretable rules.



\newpage

\vspace*{2cm}

\rule{\linewidth}{1 mm} \\[1 cm]

{\large Yo, \textbf{\theauthor}, estudiante del Grado en Ingeniería Informática de la \textbf{Escuela Técnica Superior de Ingenierías Informática y de Telecomunicación de la Universidad de Granada}, con DNI 77021623-M, autorizo la ubicación de la siguiente copia de mi Trabajo de Fin de Grado en la biblioteca del centro para que pueda ser consultada por las personas que lo deseen.}

\vspace{7cm}

Firmado: Antonio David Villegas Yeguas

\vspace{2cm}

Granada, \thedate



\newpage

\vspace*{2cm}

\rule{\linewidth}{1 mm} \\[1 cm]

D. Óscar Cordón García, profesor del Departamento de Ciencias de la Computación e Inteligencia Artificial de la Universidad de Granada.

\vspace{1cm}

D. Sergio Damas Arroyo, profesor del Departamento de Lenguajes y Sistemas Informáticos de la Universidad de Granada.

\vspace{1cm}

\textbf{Informan:}

\vspace{1cm}

Que el presente trabajo, titulado \textbf{Aprendizaje automático de un sistema interpretable de ayuda a la decisión para la estimación de la edad a partir de los huesos del pubis}, ha sido realizado bajo su supervisión por Antonio David Villegas Yeguas, y autorizamos la defensa de dicho trabajo ante el tribunal que corresponda.

\vspace{1cm}

Y para que conste, expiden y firman el presente informe en Granada a \thedate.

\vspace{5cm}

\textbf{Óscar Cordón García.}

\textbf{Sergio Damas Arroyo.}

\newpage

{\Large \textbf{Agradecimientos}}
