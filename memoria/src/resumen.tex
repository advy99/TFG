\clearpage
\begin{center}
	{\large\textbf{\thetitle}}


	\theauthor
\end{center}

\textbf{Palabras clave:} Antropología Forense, Estimación de la Edad a partir de Restos Óseos, Inteligencia Artificial Explicable,
Aprendizaje Automático, Programación Genética, Metaheurísticas.

\textbf{Resumen:}

La estimación de la edad, tanto de personas fallecidas como vivas, es una de las tareas más importantes en la antropología forense. Actualmente este proceso de estimación de la edad se basa en analizar ciertas características como la apariencia, morfología y patrones de osificación en los huesos de los individuos. Entre los distintos métodos utilizados destaca la sínfisis púbica debido a su alta fiabilidad. Este proceso se realiza de forma manual, aunque en los últimos años han aparecido diversos trabajos que intentan automatizar la estimación de la edad a partir de restos óseos. Los resultados de estos métodos son demasiado complejos para servir de apoyo a los forenses de cara a mejorar y perfeccionar la técnica, a pesar de conseguir unas estimaciones aceptables.

Por este motivo uno de los últimos enfoques aplicados sugiere buscar métodos de inteligencia artificial explicable, permitiendo obtener modelos sencillos y facilmente entendibles, aunque eso conlleve una peor estimación por parte del modelo, ya que el objetivo final no es automatizar por completo la tarea, si no servir de apoyo al forense y de esta forma facilitar su trabajo a la vez que se puede llegar a encontrar detalles que antes no se han tenido en cuenta en el proceso de estimación de la edad.

En este trabajo se partirá de la propuesta de Thomas Wingate Todd en el año 1920, en el que proponía una forma de automatizar la clasificación de restos en diez fases de edad, de cara a trabajar con el conjunto de datos de características de sínfisis púbica del laboratorio de Antropología Física de la Universidad de Granada.

Este problema se trata de un problema complejo de aprendizaje automático debido a que el tamaño del conjunto de datos es muy pequeño, tiene una gran cantidad de características y está altamente desbalanceado. Sumado a esto, tenemos como objetivo que el modelo final sea simple de cara a dar con un buen resultado que los forenses sean capaces de aplicar.

Para cumplir estos objetivos, en este trabajo se tratarán diversas técnicas para balancear los datos, reducir su dimensionalidad y se tratarán diversos enfoques de algoritmos genéticos de cara a obtener un modelo simple.

Para comenzar se intentará utilizar algoritmos basados expresiones matemáticas de cara a, con una sola fórmula, estimar la edad de cada individuo, primero explorando y entrenando el enfoque clasico de Programación Genética, mejorarlo añadiendo un algoritmo genético a la Programación Genética (GA-P) con el objetivo de suplir algunos de sus problemas, y finalmente utilizando un sistema basado en gramáticas para obtener reglas facilmente interpretables.

\newpage


\begin{center}
	{\large\textbf{\thetitleEN}}


	\theauthor
\end{center}

\textbf{Keywords:} Forensic anthropology, Age estimation from bones remains, Explainable Artificial Intelligence,
Machine Learning, Genetic Programming, Metaheuristics.

\textbf{Abstract:}

Bla bla bla bla

\newpage

\vspace*{2cm}

\rule{\linewidth}{1 mm} \\[1 cm]

{\large Yo, \textbf{\theauthor}, estudiante del Grado en Ingeniería Informática de la \textbf{Escuela Técnica Superior de Ingenierías Informática y de Telecomunicación de la Universidad de Granada}, con DNI 77021623-M, autorizo la ubicación de la siguiente copia de mi Trabajo de Fin de Grado en la biblioteca del centro para que pueda ser consultada por las personas que lo deseen.}

\vspace{7cm}

Firmado: Antonio David Villegas Yeguas

\vspace{2cm}

Granada, \thedate



\newpage

\vspace*{2cm}

\rule{\linewidth}{1 mm} \\[1 cm]

D. Óscar Cordón García, profesor del Departamento de Ciencias de la Computación e Inteligencia Artificial de la Universidad de Granada.

\vspace{1cm}

D. Sergio Damas Arroyo, profesor del Departamento de Lenguajes y Sistemas Informáticos de la Universidad de Granada.

\vspace{1cm}

\textbf{Informan:}

\vspace{1cm}

Que el presente trabajo, titulado \textbf{Aprendizaje automático de un sistema interpretable de ayuda a la decisión para la estimación de la edad a partir de los huesos del pubis}, ha sido realizado bajo su supervisión por Antonio David Villegas Yeguas, y autorizamos la defensa de dicho trabajo ante el tribunal que corresponda.

\vspace{1cm}

Y para que conste, expiden y firman el presente informe en Granada a \thedate.

\vspace{5cm}

\textbf{Óscar Cordón García.}

\textbf{Sergio Damas Arroyo.}

\newpage

{\Large \textbf{Agradecimientos}}


Bla bla bla bla
